%----------------------------------------------------------------------------------------
%	PACKAGES AND OTHER DOCUMENT CONFIGURATIONS
%----------------------------------------------------------------------------------------
\documentclass[paper=a4, fontsize=11pt]{scrartcl} % A4 paper and 11pt font size
\usepackage[T1]{fontenc} % Use 8-bit encoding that has 256 glyphs
\usepackage{fourier} % Use the Adobe Utopia font for the document - comment this line to return to the LaTeX default
\usepackage[english]{babel} % English language/hyphenation
\usepackage{amsmath,amsfonts,amsthm} % Math packages
\usepackage{lipsum} % Used for inserting dummy 'Lorem ipsum' text into the template
\usepackage{sectsty} % Allows customizing section commands
\allsectionsfont{\centering \normalfont\scshape} % Make all sections centered, the default font and small caps
\usepackage{fancyhdr} % Custom headers and footers
\usepackage{amsmath}
\usepackage{graphics}
\usepackage{graphicx}

\pagestyle{fancyplain} % Makes all pages in the document conform to the custom headers and footers
\fancyhead{} % No page header - if you want one, create it in the same way as the footers below
\fancyfoot[L]{} % Empty left footer
\fancyfoot[C]{} % Empty center footer
\fancyfoot[R]{\thepage} % Page numbering for right footer
\renewcommand{\headrulewidth}{0pt} % Remove header underlines
\renewcommand{\footrulewidth}{0pt} % Remove footer underlines
\setlength{\headheight}{13.6pt} % Customize the height of the header

\numberwithin{equation}{section} % Number equations within sections (i.e. 1.1, 1.2, 2.1, 2.2 instead of 1, 2, 3, 4)
\numberwithin{figure}{section} % Number figures within sections (i.e. 1.1, 1.2, 2.1, 2.2 instead of 1, 2, 3, 4)
\numberwithin{table}{section} % Number tables within sections (i.e. 1.1, 1.2, 2.1, 2.2 instead of 1, 2, 3, 4)

\setlength\parindent{0pt} % Removes all indentation from paragraphs - comment this line for an assignment with lots of text

%----------------------------------------------------------------------------------------
%	TITLE SECTION
%----------------------------------------------------------------------------------------

\newcommand{\horrule}[1]{\rule{\linewidth}{#1}} % Create horizontal rule command with 1 argument of height

\title{	
\normalfont \normalsize 
\horrule{0.5pt} \\[0.4cm] % Thin top horizontal rule
\huge CS 760 Homework 3: Neural Network\\ % The assignment title
\horrule{2pt} \\[0.5cm] % Thick bottom horizontal rule
}

\author{Qihong Lu} % Your name
\date{\normalsize\today} % Today's date or a custom date

\begin{document}

\maketitle % Print the title

%----------------------------------------------------------------------------------------
%	PROBLEM 1
%----------------------------------------------------------------------------------------

\section*{Question1}

To run the program, type the following command: 
$$ \text{nnet l h e <train-set-file> <test-set-file>} $$ 
where l specifies the learning rate, h the number of hidden units and e the number of training epochs. \\\\\\

Code: 
\begin{itemize}
	\item nn\_alg.py: implements binary classification network with zero or one hidden layer 
	\item util.py: the definitions of some constants and helper functions 
	\item nnet.py: run neural network, return test results
\end{itemize}
 
\hfill 

Dependencies: 
\begin{itemize}
  \item pyhton 2.7 
  \item numpy 
  \item scipy 
  \item sys
\end{itemize}



%----------------------------------------------------------------------------------------
%	PROBLEM 2
%----------------------------------------------------------------------------------------
\newpage
\section*{Question2}
\textbf{For this part, you will explore the effect of hidden units and the number of training epochs. Using heart\_train.arff and heart\_test.arff, you should make two graphs showing error-rates versus the number of training epochs. For the first graph, plot training and testing error rates for a single-layer network trained for 1, 10, 100 and 500 epochs, using a learning rate of 0.1. For the second graph, plot similar curves for a network with 20 hidden units. Be sure to label the axes of your plots.}
\begin{center}
\includegraphics[scale=.5]{pics/learningCurve_1.png}
\end{center}
\begin{center}
\includegraphics[scale=.5]{pics/learningCurve_1.png}
\end{center}

%----------------------------------------------------------------------------------------
%	PROBLEM 3
%----------------------------------------------------------------------------------------
\newpage
\section*{Question3}
\textbf{For this part, you should produce ROC curves for two data sets. Use the activation of the output unit as the measure of confidence that a given test instance is positive, and plot ROC curves for both the heart data set indicated above, and the lymphography data set lymph\_train.arff, lymph\_test.arff. Be sure to label the axes of your plots.}

\begin{center}
\includegraphics[scale=.5]{pics/roc.png}
\end{center}


\end{document}