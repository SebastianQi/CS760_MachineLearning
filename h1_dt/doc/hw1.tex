%----------------------------------------------------------------------------------------
%	PACKAGES AND OTHER DOCUMENT CONFIGURATIONS
%----------------------------------------------------------------------------------------
\documentclass[paper=a4, fontsize=11pt]{scrartcl} % A4 paper and 11pt font size
\usepackage[T1]{fontenc} % Use 8-bit encoding that has 256 glyphs
\usepackage{fourier} % Use the Adobe Utopia font for the document - comment this line to return to the LaTeX default
\usepackage[english]{babel} % English language/hyphenation
\usepackage{amsmath,amsfonts,amsthm} % Math packages
\usepackage{lipsum} % Used for inserting dummy 'Lorem ipsum' text into the template
\usepackage{sectsty} % Allows customizing section commands
\allsectionsfont{\centering \normalfont\scshape} % Make all sections centered, the default font and small caps
\usepackage{fancyhdr} % Custom headers and footers
\usepackage{amsmath}
\usepackage{graphics}
\usepackage{graphicx}

\pagestyle{fancyplain} % Makes all pages in the document conform to the custom headers and footers
\fancyhead{} % No page header - if you want one, create it in the same way as the footers below
\fancyfoot[L]{} % Empty left footer
\fancyfoot[C]{} % Empty center footer
\fancyfoot[R]{\thepage} % Page numbering for right footer
\renewcommand{\headrulewidth}{0pt} % Remove header underlines
\renewcommand{\footrulewidth}{0pt} % Remove footer underlines
\setlength{\headheight}{13.6pt} % Customize the height of the header

\numberwithin{equation}{section} % Number equations within sections (i.e. 1.1, 1.2, 2.1, 2.2 instead of 1, 2, 3, 4)
\numberwithin{figure}{section} % Number figures within sections (i.e. 1.1, 1.2, 2.1, 2.2 instead of 1, 2, 3, 4)
\numberwithin{table}{section} % Number tables within sections (i.e. 1.1, 1.2, 2.1, 2.2 instead of 1, 2, 3, 4)

\setlength\parindent{0pt} % Removes all indentation from paragraphs - comment this line for an assignment with lots of text

%----------------------------------------------------------------------------------------
%	TITLE SECTION
%----------------------------------------------------------------------------------------

\newcommand{\horrule}[1]{\rule{\linewidth}{#1}} % Create horizontal rule command with 1 argument of height

\title{	
\normalfont \normalsize 
\horrule{0.5pt} \\[0.4cm] % Thin top horizontal rule
\huge CS760 Homework 1: Decision Tree\\ % The assignment title
\horrule{2pt} \\[0.5cm] % Thick bottom horizontal rule
}

\author{Qihong Lu} % Your name

\begin{document}

\maketitle % Print the title


\section*{Question1}
Code: 
\begin{itemize}
  \item dt-learn.py: the main program that implements the ID3  decision tree algorihtm 
  \item util.py: the definitions of some constants
  \item decisionTreeNode.py: the definition of the tree node class 
\end{itemize}
 
\hfill 

Dependencies: 
\begin{itemize}
  \item pyhton 2.7 
  \item numpy 
  \item scipy 
  \item sys
\end{itemize}


\section*{Question2}


\begin{center}
\includegraphics[scale=.5]{pics/heart_acc_data.png}
\end{center}
\begin{center}
\includegraphics[scale=.5]{pics/diabetes_acc_data.png}
\end{center}

Here are the learning curve: the performance,  measured in terms of classification accuracy,  against the amount of training data used when constructing the decision tree. In general, the performance is increasing as more training data were supplied. This trend holds for both of the data sets we explored. 

\newpage



%----------------------------------------------------------------------------------------
%	PROBLEM 3
%----------------------------------------------------------------------------------------

\section*{Question3}
\begin{center}
\includegraphics[scale=.5]{pics/heart_acc_m.png}
\end{center}
\begin{center}
\includegraphics[scale=.5]{pics/diabetes_acc_m.png}
\end{center}

As we increase m, the classification accuracy increases for both data sets. It shows that, as we had no pruning, the decision tree tends to overfit the training data. And by early stopping (increase m), we obtain a simpler tree which tends to ameliorate the overfitting issue. 

\end{document}